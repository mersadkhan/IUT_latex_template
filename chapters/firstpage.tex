\university{دانشگاه صنعتی اصفهان}
\department{دانشکده فیزیک}
\type{پایان‌نامه کارشناسی ارشد}
\degree{کارشناسی ارشد}
\subject{فیزیک ماده چگال }
\field{محاسباتی}
\title{
راهنمای لاتک برای نوشتن پایان‌نامه در دانشگاه صنعتی اصفهان
}
% اگر عنوان پایان‌نامه شما طولانی است می‌توانید بخشی از آن را در قسمت زیر وارد کنید. 
\tit{}
\supervisor{دکتر محمود اشرفی‌زاده}
% چنانچه استاد راهنمای دوم دارید نام ایشان را در داخل گیومه متغییر زیر بنویسید
\secsupervisor{دکتر سیَد جواد هاشمی‌فر}
\advisor{دکتر مجتبی اعلایی} 
% چنانچه استاد راهنمای دوم دارید متغییر زیر را ست نمایید
\secadvisor{} 
% نام استاد(دان) مشاور را وارد کنید. چنانچه استاد مشاور ندارید، دستورات پایین را غیرفعال کنید.
\defenddate{1397/2/4}
\author{مرصاد مستقیمی}
\supervisorM{آقای دکتر محمود اشرفی‌زاده}
%استاد مشاور دوم می تواند خالی باشد
\secsupervisorM{آقای دکتر سیّد جواد هاشمی‌فر}
\advisorM{آقای دکتر مجتبی اعلایی}
%استاد راهنمای دوم می تواند خالی باشد
\secadvisorM{}
\refreeinB{آقای دکتر اسماعیل عبدالحسینی سارسری}
\refreeinA{خانم دکتر نفیسه رضایی}
%ممتحنین خارجی می تواند خالی باشد
 \refreeoutA{}%
\refreeoutB{}%
\DGC{آقای دکتر فرهاد شهبازی}
\thesisdate{اردیبهشت ۱۳۹۷}
%
\makefatitle
\approval
