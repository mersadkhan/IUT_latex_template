\chapter{مدیریت مراجع درتک}\label{App:Refan}
\thispagestyle{empty}
\chapter{اتصال بسته محاسباتی \lr{FPLO} و\lr{Wannier90}}\label{chp:chap4}
\thispagestyle{empty}
\rhead{\leftmark}
%================================================================================

\section{چگالی‌ حالات هیبریدی و هیبریدی منطقه ای }\label{seq:4.2}
چگالی حالات کل الکترون‌ها در یک ساختار از عبارت زیر حاصل می‌شود:\cite{Martin2004}
\begin{equation}
 \rho(\varepsilon)=\frac{1}{N_{{\bf k}}}\sum_{n,{\bf k}}\delta(\varepsilon_{n,{\bf k}}-\varepsilon)=\frac{V_{cell}}{(2\pi)^d} \int_{BZ} d{\bf k}\delta(\varepsilon_{n,{\bf k}}-\varepsilon)
\end{equation}
که با نوشتار دیراک 
\begin{equation}
 \rho(\varepsilon)=\sum_{n}\langle\psi_n|\psi_n\rangle\delta(\varepsilon_{n}-\varepsilon)
\end{equation}
هم ارز است  و در آن $\varepsilon_n$ ویژه مقدار ویژه حالت $\psi_n$ است. با استفاده از پایه‌های کامل و بهنجار $|i\rangle\langle i|=1$ و $|r\rangle\langle r|=1$ می‌توانیم عبارات زیر را بنویسیم.
\begin{align}
\rho_{i}(\varepsilon)=\sum_n\langle\psi_n|i\rangle\langle i|\psi_n\rangle\delta(\varepsilon-\varepsilon_n)\label{eq:4.2.3}\\
\rho(r,\varepsilon)=\sum_n\langle\psi_n|r\rangle\langle r|\psi_n\rangle\delta(\varepsilon-\varepsilon_n)\label{eq:4.2.4}
\end{align}
عبارات فوق به ترتیب \glspl{چگالی حالات تصویر شده}و \glspl{چگالی حالات موضعی}هستند.\cite{gpaw} 
اربیتال‌های $|i\rangle$ و $|r\rangle$ می‌توانند هر اربیتال دلخواه انتخابی- با رعایت شرط کاملیّت و بهنجار 
بودن- باشند؛ امّا ما در محاسبات خود در برنامه \lr{FPLO2WANNIER} از بخش زاویه‌ای توابع $g_n$ که در 
بخش \ref{seq:4.1} معرفی شده‌اند یعنی $\Theta_{lm_{\mathrm{r}}}(\theta,\varphi)$های معرفی شده در جداول 
\ref{tab:angular} و \ref{tab:hybrids} به عنوان توابع انتخابی استفاده نموده‌ایم. لازم به یادآوری است که 
جدول \ref{tab:angular} اربیتالهای اتمی و جدول  \ref{tab:hybrids} اربیتال‌های هیبریدی را به دست می‌دهد و 
تنها بخش حقیقی آنها را شامل می‌شود؛\cite{Rehfeld1978,Weissteina} لذا شرط به هنجارش در استفاده از آنها 
رعایت نشده است و نباید انتظار داشت که انتگرال حاصل از چگالی حالات آنها با تعداد الکترون‌های مورد بررسی 
برابر باشد؛ البته برای یک حالت خاص و در یک فایل جداگانه این مورد با توابع به‌هنجار نیز توسط 
\lr{FPLO2WANNIER} محاسبه‌می‌گردد.

در کدهای محاسباتی معمولا چگالی حالات موضعی به دست‌می‌آید و همچنین چگالی حالات تصویر شده  برای 
اربیتال‌های اتمی \lr{s,p,f} و \lr{f} به دست می‌آید. توجه به این نکته ضرری است که در کدهای محاسباتی شمارش 
حد بالا و پایین انتگرال‌ها در چگالی حالات تصویر شده بر مبنای تسلسل انرژی اربیتال‌های اتم‌های شرکت‌کننده در 
ساختار است. این توالی مشخص می‌کند که کدام نوارهای انرژی باید برای تصویر شدن روی اربیتال اتمی مورد نظر 
\lr{s,p,d}
انتخاب شوند و کدام یک در چگالی حالت آن اربیتال نقشی ندارند.(جدول ترتیب پرشدن اربیتالها \ref{wikiatomic}) 
چون در کد نوشته شده در این پایان‌نامه این قاعده هنوز وارد نشده است -چون کار زمانبری است و زمان کافی بعد 
از تغییر جهت به این سمت وجود نداشت- هم اکنون تمام نوارها روی اربیتال انتخابی ما تصویر می‌شوند که این باعث 
می‌شود صرفاً چگالی حالات کل نسبت  به اربیتال انتخابی تغییر کند و نوارهای مربوط به زیرلایه‌های غیر از 
زیرلایه‌های مطلوب نیز در این محاسبات دخیل باشند. البته‌می‌توان با انجام محاسبات صرفاً برای یک اربیتال 
انتخابی خاص و انتخاب نوارهای متعلق به آن به نتیجه‌نسبتاً مطلوبی رسید. از طرفی چون در توابع هیبریدی تمام 
الکترون‌های چند اربیتال مختلف ممکن است دخیل باشند امکان انتخاب نوارهای انرژی و اربیتال‌های انتخابی هیبریدی 
مرتبط، امکان بررسی چگالی حالات هیبریدی را به دست می‌دهد که همان نتیجه مطلوب ما بوده و در ادامه نتایج آن 
خواهد آمد. به علاوه با ایجاد امکان انتقال مراکز اربیتالهای انتخابی به نقطه‌ای خاص -مثلاً روی اتمی خاص و یا 
روی پیوندی خاص- امکان به دست آوردن چگالی حالات موضعی با توجّه به شرایط فوق توسط کد \lr{FPLO2WANNIER} محیا 
شده است.
% \begin{figure}[ht]
% \centering
% \includegraphics[scale=0.6]{Electron_orbitals}
% \caption{\label{fig:4.2.1}
% توالی انرژی برای زیرلایه‌ها بر اساس اعداد کوانتومی آنها و نمونه ای از اربیتاهای اتمی و هیبریدی\cite{wikiatomic} }
% \end{figure}
\begin{table}
\renewcommand{\arraystretch}{0.4}
\begin{center}
 \begin{tabular}{|c||c|c|c|c|c|c|}
 \hline
$n$  & s & p & d & f & g & h \\
\hline\hline
1     & 1&  &  &  &  &   \\
\hline
2    & 2 &3 &   &  &  &  \\
\hline
3     & 4 &5 & 7 &  &  &  \\
\hline
4     & 6 & 8 & 10 & 13 &  &  \\
\hline
5     & 9 & 11 & 14 & 17 &21 &  \\
\hline
6     &16 & 19 & 23 & 27 & 32 & 37 \\
\hline
\end{tabular}
\caption{ 
توالی انرژی برای زیرلایه‌ها بر اساس اعداد کوانتومی آنها. شماره لایه\lr{(n)} در ستون اوّل و هر خانه مشخص‌کننده ترتیب الکترون‌گیری اربیتال مربوطه است. تعداد الکترونها اختصاص یافته به هر اربیتال وابسته به محاسبه و مواد شرکت کننده در محاسبه است.\cite{wikiatomic}
}
\end{center}
\end{table}
ترسیم چگالی حالات با استفاده از فرمولهای فوق به خاطر ماهیّت تابع دیراک به صورت خطوط تیز است. برای درک بهتر 
چگالی حالات معمولاً از یک پهن‌شدگی برای نمایش چگالی حالات استفاده‌می‌شود. در 
\lr{FPLO2WANNIER}
نیز از تابع گاوسی به این منظور استفاده شده است. به علاوه فایلی جداگانه همان محاسبه‌ی فرمولهای فوق 
را نیز به دست‌ می‌دهد. چگالی حالات با ورود پهن‌شدگی به شکل زیر خواهد شد.
\begin{align}
\rho_{i}(\varepsilon)=\sum_n\frac{1}{\sigma\sqrt{\pi}}\exp(-\frac{(\varepsilon-\varepsilon_i({\bf k}))^2}{\sigma^2})\langle\psi_n|i\rangle\langle i|\psi_n\rangle\delta(\varepsilon-\varepsilon_n)\label{eq:4.2.5}\\
\rho(r,\varepsilon)=\sum_n\frac{1}{\sigma\sqrt{\pi}}\exp(-\frac{(\varepsilon-\varepsilon_i({\bf k}))^2}{\sigma^2})\langle\psi_n|r\rangle\langle r|\psi_n\rangle\delta(\varepsilon-\varepsilon_n)\label{eq:4.2.6}
\end{align}
مقدار $\sigma$به صورت پیش فرض برابر $0.2$ الکترون‌ولت در نظر گرفته شده است که کاربر‌می‌تواند آن را تغییر 
دهد.
%===================================================================================
\section{معرفی ورودی‌ها و خروجی‌ها}\label{seq:4.3}
به طور کلی ورودی برنامه \lr{FPLO2WANNIER} به شرح زیر است.
\begin{itemize}
 \item\lr{{\bf +symmetry}}و یا \lr{{\bf +symminfo}}
 که هردو به صورت خودکار توسط 
 \lr{FPLO}
 و
 \lr{Fedit}
  تولید و شامل نگاشت‌های مربوط به نقاط هم ارز در منطقه اوّل بریلوئن است و در انتهای 
آن بردارهای شبکه و شبکه وارون و تبدیل‌های آنها آمده است که این قسمت توسط برنامه \lr{FPLO2WANNIER}  
استخراج‌ می‌شود و در خروجی نیز چاپ می‌گردد.
 \item
   \lr{{\bf +gridscfpsi.g$\bf 001$}}
 فایل تابع موج که توسط زیر پنجره \lr{grid} از رابط کاربری \lr{fedit} تنظیمات مشبندی آن انجام‌می‌گیرد 
همچنین می‌توان دامنه انرژی در این قسمت تعیین کرد و باید توجه داشت که ابتدای این دامنه در یک گاف انرژی 
بوده و با دامنه‌ی انرژی که در زیرپنجره‌ی \lr{bandplot} تعین می‌شود یکسان باشد. نام آن اختیاری است و در 
توضیحات مربوط به این فایل در دستورالعمل \lr{FPLO} آمده است.\cite{Koepernik2009}
 \item\lr{{\bf wannier.wout}}
 که از \glspl{محاسبات مقدماتی} بسته \lr{wannier90}  به دست می‌آید. از این فایل بردارهای همسایگی اوّل استخراج‌می‌شود. \lr{wannier} یک نام پیش‌فرض است.
 \item\lr{{\bf =.kp}}
 که از آن نقاط مشبندی منطقه اوّل بریلوئن استخراج ‌می‌شود. در ابتدای پروژه این فایل استفاده ‌می‌شد ولی اکنون می‌توان با تغییراتی نیاز به این فایل را حذف نمود که 
این کار هنوز انجام نشده است. البته این فایل برای استخراج ویژه مقادیر انرژی در نقاط مختلف شبکه وارون نیز استفاده می‌شود؛ فلذا بایستی آن را متناسب با مش‌بندی 
شبکه وارون و با استفاده از برنامه‌ی جنبی \lr{kmeshfplo} ایجاد نمود.\ref{kmeshfplo}
 \item\lr{{\bf +band\_kp}}
ویژه مقادیر روی مشبندی شبکه وارون را که در فایل \lr{=.kp} آمده به دست‌می‌دهد و برای استخراج آن باید تنظیماتی در زیرپنجره‌ی \lr{bandplot} در \lr{Fedit} 
انجام داد.
 فایل \lr{wannier.win} که حامل اطلاعات مربوط به ورودی اجرای برنامه \lr{wannier90} به اضافه‌ی کلیدهای اضافی است که ورودی‌های برنامه \lr{FPLO2WANNIER} می‌باشند. ورودی \lr{wannier.win} اجرای \lr{FPLO} از روی این فایل ساخته می‌شود.
\end{itemize}
در فایل \lr{wannier.win} به جز آنچه در دستورالعمل بسته محاسباتی \lr{wannier90} آمده است گزینه‌های زیر قابل انتخاب است.\cite{Wannier902013} انتخاب‌های زیر صرفاً در هنگام اجرای \lr{FPLO2WANNIER} کاربرد دارند و برای اجراهای مربوط به \lr{wannier90} بدون کاربرد بوده و باید غیرفعال باشند که در این زمینه \lr{FPLO2WANNIER} هم مانند \lr{wannier90} از علامت‌های ! و \# استفاده می‌کند.
\begin{itemize}
 \item\lr{{\bf special\_bands}}
شماره نوارهایی که می‌خواهیم روی آنها محاسبات \lr{FPLO2WANNIER} انجام شود در مقابل این کلید وارد می‌شوند. این شماره‌ها بر اساس شماره‌های به دست آمده از 
شماره ستون نوارهای فایل\lr{+band\_kp} است؛ به این ترتیب که ستون اول در این فایل نقاط منطقه اوّل و ستون‌های بعدی مربوط به نوار اول تا \lr{n}ام هستند. می‌توان 
دامنه‌ای از نوارها را با - مشخص نمود. به عنوان مثال 
 \begin{latin}
\begin{lstlisting}[style=Mybash]
special_bands 1,2,5-7
\end{lstlisting}
\end{latin}
نوارهای ۱،۲،۵،۶و۷ را از سایر نوارها جدا می‌کند.
 
 \item\lr{{\bf energy\_dos}}
شامل سه عدد که به ترتیب ابتدا، انتها ،و تعداد بازه‌های انرژی برای چگالی حالات تصویر را‌می‌گیرد.
\begin{latin}
\begin{lstlisting}[style=Mybash]
energy_dos -15 0 1000
\end{lstlisting}
\end{latin}
در این مثال بازه انرژی $−۱۵$ تا $۰$ الکترون‌ولت به $۱۰۰۰$ قسمت تقسیم‌می‌شود تا چگالی حالات روی آنها حساب شود.
\item\lr{{\bf dos\_sigma}}
می‌تواند مقدار پهن‌شدگی تابع گاوسی در محاسبه چگالی حالات را تغییر دهد.
\begin{latin}
\begin{lstlisting}[style=Mybash]
dos_sigma 0.2
\end{lstlisting}
\end{latin}
\item\lr{{\bf exclude\_bands}}
شماره نوارهایی که در این جا آمده باشند از لیست محاسبه حذف می‌شوند. در حال حاضر در کد پس از حذف این نوارها برچسب دهی به نوارهای باقی مانده به ترتیب خواهد بود. به 
عبارتی اگر ده نوار داشته باشیم و نوار ۵و۹ را حذف کنیم نوارهای از یک تا ۸ برچسب‌می‌خورند و ساختار به این شکل محاسبه‌می‌شود. درستی این کار هنوز بررسی نشده است اما 
این گزینه در بسته \lr{wannier90} وجود دارد. قاعده دامنه نوارها که قبلاً برای \lr{special\_bands} بیان شد در این مورد نیز صادق است.
\item\lr{{\bf dos\_projections}}
بخش تصویرهای چگالی حالات به شکل زیر تعریف‌می‌شود و نوع تابع هیبریدی را که می‌خواهیم تصویر را روی آن مشخص کنیم به دست می‌دهد. می‌توان یک تقریب از چگالی حالت 
منطقه‌ای نیز با وارد کردن مختصات به دست آورد. دقّت شود که برای به دست آوردن توابع تصویر باید در قسمت \lr{special\_bands} به تعداد الکترون‌هایی که اربیتال 
مذبور در کل شرکت‌می‌دهد توجه داشت و همان نوارها رانیز مشخص نمود.

\begin{latin}
\begin{lstlisting}[style=Mybash]
  begin dos_projections
  c=0.0,0.0,0.0:s
 end dos_projections
\end{lstlisting}
\end{latin}
و یا 
\begin{latin}
\begin{lstlisting}[style=Mybash]
  begin dos_projections
  c=0.0,0.0,0.0:dx2-y2
 end dos_projections
\end{lstlisting}
\end{latin}
هر دوی این موارد در مبدأ بررسی شده ‌اند. این که خود کد بر اساس تعداد الکترون‌های ظرفیت مشخص کند کدام نوار مربوط به کدام اربیتال است خود یک پروژه کدنویسی است که 
فرصت آن در این پایان‌نامه پیش نیامد و اکنون فقط می‌توان از طریق فایل ورودی \lr{wannier.win} نوارهای اربیتال مذبور را مشخص نمود تا کد تصویر چگالی حاالت روی حالت 
هیبریدی دلخواه را محاسبه کند.
\item{\bf توابع حدس گاوسی}
به شکل زیر تعریف‌می‌شوند و از تابع گاوسی برای تولید یک فایل \lr{wannierfr.amn} استفاده می‌کنند. 
\begin{latin}
\begin{lstlisting}[style=Mybash]
frprojections c=0,0,0:sigmafr=9
\end{lstlisting}
\end{latin}
 عدد \lr{sigmafr} مقدار $\sigma$ را در معادله گاوس مشخص‌می‌کند. \cite{Weisstein}
 \begin{equation}
  \sigma(\varepsilon-\varepsilon_i({\bf k}))\longrightarrow\frac{1}{\sigma\sqrt{\pi}}\exp(-\frac{(\varepsilon-\varepsilon_i({\bf k}))^2}{\sigma^2})
 \end{equation}
 این توابع بیشتر برای امتحان‌کردن کد با توابع اوّلیّه مختلف ایجاد شده‌اند.
\end{itemize}
خروجی‌های برنامه \lr{FPLO2WANNIER} به صورت پیشفرض در یک پوشه به نامه \lr{FPLO2WAN} تولید‌می‌شوند. فایل‌های زیر بعد از اجرا در آن به وجود‌می‌آیند.
\begin{itemize}
 \item\lr{{\bf wannier.win}}
 بازنویسی فایل ورودی با همین نام بدون کلید‌های مربوط به برنامه \lr{FPLO2WANNIER} است تا اجرای \lr{wannier90} بدون خطا باشد.
 \item\lr{{\bf wannier.mmn}}
 حاوی ماتریس $M_{mn}^{(\bf{k,b})}$ یعنی مجموعه انتگرال‌های‌های بخش دوره‌ای تابع موج و همسایگی‌های اوّل آن‌ها 
می‌شود که در معادله‌ی \ref{eq:2.6.13} معرفی شده است. این فایل با استخراج توابع موج از بسته محاسباتی \lr{FPLO} تولید و بررسی شد. برای این منظور ابتدا لازم بود 
بهنجار بودن توابع موج خروجی \lr{FPLO} بررسی شود. پس از مطالعه و کدنویسی‌های فراوان و استفاده از دو روش مختلف انتگرال‌گیری -جمع روی توابع و \glspl{انتگرال‌گیری 
سه‌خطی}\cite{trilinear}- که جواب‌های یکسانی ارائه کرده‌اند توانستیم فایلی بزرگ -در حدود چند ده گیگابایت!که همان فایلهای خروجی \lr{FPLO}  با عنوان 
\lr{+gridscfpsi.g$001$}
است- استخراج کنیم که محتوای آن توابع موج به هنجار در تمامی نقاط منطقه اوّل بریلوئن است. چون این ماتریس از توابع موج استخراج می‌شود؛ لذا از انتگرال زیر برای به 
دست آوردن آن استفاده شده است.\cite{Marzari1997}
  \begin{equation}
 M_{mn}^{(\bf{k,b})}=exp(-i\textbf{b.r})\langle\psi_{m{\bf k}}|\psi_{n{\bf k}+{\bf b}}\rangle
 \end{equation}
 \item\lr{{\bf wannier.amn}}
 حاوی توابع تصویر تولید شده یا همان ماتریس $(A_{\mathbf{k}})_{mn}$ است که در معادلات بخش \ref{sec:sec2.4} به آن اشاره شد. در این فایل از ماتریس  
$(A_{\mathbf{k}})_{mn}$ 
به شکل زیر استفاده شده است که تمامی فضا -و نه $\frac{1}{8}$ مشخص شده توسط مشبندی شبکه مستقیم- را به شکل زیر می‌پوشاند.
%  \begin{latin}
 \begin{align}\label{eq:4.3.3}
  (A_{{\mathbf k}})_{mn}=& \int_{0}^{a_{1}}\int_{0}^{a_{2}}\int_{0}^{a_{3}}\psi_{m{\bf k}}(x,y,z) \{ g_{n}(x,y,z)\nonumber\\
  & +  g_{n}(x-a_{1},y,z)e^{ik_{1}a_{1}}  +  g_{n}(x,y-a_{2},z)e^{ik_{2}a_{2}}\nonumber\\
  &  +  g_{n}(x,y,z-a_{3})e^{ik_{3}a_{3}}  +  g_{n}(x-a_{1},y-a_{2},z)e^{ik_{1}a_{1}}e^{ik_{2}a_{2}}\nonumber\\
  &  +  g_{n}(x-a_{1},y,z-a_{3})e^{ik_{1}a_{1}}e^{ik_{3}a_{3}}+  g_{n}(x,y-a_{2},z-a_{3})e^{ik_{1}a_{1}}e^{ik_{1}a_{1}}\nonumber\\
  &  +  g_{n}(x-a_{1},y-a_{2},z-a_{3})e^{ik_{1}a_{1}}e^{ik_{2}a_{2}}e^{ik_{3}a_{3}} \}
 \end{align}
%  \end{latin}
  \item {\lr{\bf wannier\_$1$.amn}}
  همان ماتریس $(A_{\mathbf{k}})_{mn}$که فقط همان مشبندی $\frac{1}{8}$ فضایی را داراست.
  \item\lr{{\bf xsf}}
   اگر برنامه \lr{FPLO2WANNIER} به شکل
\begin{latin}
\begin{lstlisting}[style=Mybash]
 ~$ fplo2wannier -x y
\end{lstlisting}
\end{latin}
اجرا شود این فایل‌ها که نمایش $g_n$های اوّلیّه هستند و‌می‌توان آنها را با برنامه \lr{xcrysden} مشاهده 
نمود تولید‌می‌گردد. همچنین با این اجرا از برنامه فایلهای \lr{cord} نیز تولید می‌شوند. به علاوه این 
فایل‌ها فقط نمایش $\frac{1}{8}$ فضای مش‌بندی شده است ولی یک نمونه آزمایشی از ترسیم این توابع به طور کامل 
در فایهای \lr{wannier.xsf} نیز تولید می‌شود که نمایش کاملی از شکل اربیتال‌های حدس به کار رفته در محاسبات 
را نشان می‌دهد.
\begin{figure}[ht]
 \centering
 \includegraphics[scale=0.7]{px}
\caption{\label{fig:px}
چگالی حالت اربیتال \lr{s} محاسبه شده با \lr{FPLO} و \lr{FPLO2WANNIER}
}
\end{figure}
\item\lr{{\bf cord}}
مختصات تغییر یافته برای توابع تصویر را‌می‌دهد. هم مشبندی دکارتی و هم کروی و هم تابع تصویر مربوطه در این فایل می‌آید.
\item\lr{{\bf wannier.pds}}
چگالی حالات هیبریدی بدون پهن شدگی برای تابع  $g_n$ و نوارهای انتخاب شده در را ورودی‌ می‌دهد. چنانچه انتقالی نیز وجود داشته باشد چگالی حالات هیبریدی منطقه‌ای را 
به دست خواهد داد. 
\item\lr{{\bf wannier.pds\_sigma}}
چگالی حالات فوق الذّکر که با توزیع گاوسی به آن یک پهن‌شدگی داده شده است را به دست می‌دهد
\item\lr{{\bf wannier.pds\_sigma\_orth}}
چگالی حالات فوق الذّکر که با توزیع گاوسی به آن یک پهن‌شدگی داده شده است را به دست می‌دهد. تفاوت بین این فایل و فایل قبل در 
نوع $g_n$ به کار رفته است که تنها منطقه مشبندی ارائه شده توسط \lr{FPLO} را شامل می‌شود. به عبارت دیگر 
$g_n$
در این مورد تنها جمله اول داخل آکولاد عبارت \ref{eq:4.3.3} را در بر می‌گیرد. به علاوه این تابع 
بهنجار شده و سپس محاسبات انجام‌می‌پذیرد. (در حال امتحان است.)
\end{itemize}
به غیر از پوشه‌ی \lr{FPLO2WAN}پوشه دیگری نیز تولید‌می‌شود که به صورت پیش‌فرض \lr{TEMP} نام‌گذاری شده است.و به بخش‌های زیر تقسیم‌می‌شود:
\begin{itemize}
 \item\lr{{\bf waves}}
 شامل فایل‌هایی به تعداد تمام نقاط \lr{k} -مثلاً تعدا $16\times1\times1$ عدد فایل- می‌باشد که توابع موج بر روی مش‌بندی بنیادی در همه نوارهای دلخواه که در 
ورودی شماره آنها را وارد شده است و از فایل حجیم  \lr{+gridscfpsi.g$001$} استخراج می‌شوند. این کار 
امکان دسترسی به بخش‌های مطلوب تابع موج با بارگذاری باینری آنها را فراهم‌می‌آورد و به محاسبات سرعت می‌بخشد. به علاوه چنانچه نیاز برای محاسبه بر 
روی تعداد دیگری از توابع هیبریدی وجود داشته باشد از قطعه قطعه نمودن مجدد فایل سنگین تابع موج اجتناب می کند و می توان فایلهای تقسیم شده را با استفاده از دستور 
زیر در هنگام اجرای برنامه به عنوان ورودی وارد نمود.
\begin{latin}
\begin{lstlisting}[style=Mybash]
 ~$ fplo2wannier -t TEMP
\end{lstlisting}
\end{latin}
 \item\lr{{\bf points}}
 حاوی مشبندی دکارتی شبکه می‌باشد که توابع موج در آن نقاط داده شده است. این نقاط از تقسیم‌بندی بردارهای شبکه به تعداد بخش‌های وارد شده در زیرمنوی \lr{grid} در 
رابط \lr{Fedit} ایجاد ‌می‌شوند. این مش‌بندی باید آنقدر زیاد باشد تا توابع موج ایجاد شده در فرایند بهنجارش انتگرالی برابر یک داشته باشند که مسأله‌ی بهنجارش در 
هنگام خواند توابع موج توسط \lr{FPLO2WANNIER} بررسی شده و چنانچه این توابع بهنجار نباشند با ارائه‌ی خطا از برنامه خارج می‌شود.
 \item\lr{{\bf data}}
که حاوی اطلاعات به دست آمده از خوانش فایل‌های خروجی \lr{FPLO} بوده و برای اینکه کاربر مجبور نشود در هر بار محاسبه یک بار فایل تابع موج را قطعه قطعه کند ایجاد 
شده است. چنانچه در توسعه‌های بعدی خواهان اضافه کردن بخشی برای قطع و ادامه محاسبات باشیم وجود این فایل کمک‌می‌کند. این فایل حاوی اکثر متغیر‌های به کار رفته در 
برنامه نیز‌می‌باشد که با استفاده از ابزار \glspl{لغت‌نامه} در زبان برنامه نویسی پایتون که برنامه \lr{FPLO2WANNIER} با آن نوشته شده است ایجاد و خوانده ‌می‌شود.
\end{itemize}