% !TEX TS-program = XeLaTeX
% !TeX root=main.tex
% دستورات زیر باید در اولین فایل پیوست باشند. آنها را حذف نکنید!    
\chapter{نمونه‌هایی برای وارد کردن کد برنامه‌ها}\label{App:RefMan1}
\thispagestyle{empty}
%==================================================================
نمونه کد \lr{C}
\begin{latin}
\begin{lstlisting}[style=CStyle]
#include <stdio.h>
int main(void)
{
   printf("Hello World!"); 
}
\end{lstlisting}
\end{latin}
نمونه کد بش:
\begin{latin}
\begin{lstlisting}[style=Mybash]
 ~$ sudo apt-get update & sudo apt-get upgrade
 ~$ sudo apt-get install python3 python3-numpy python3-scipy
 ~$ chmod +x fplo2wannier
\end{lstlisting}
\end{latin}
نمونه کد پایتون:
\begin{latin}
\begin{lstlisting}[style=Mypython]
 #!/usr/bin/python3 -u
from sys import argv
arg=[int(x) for x in argv[1:4]]
xtel=1.0/arg[0]
ytel=1.0/arg[1]
ztel=1.0/arg[2]
x,y,z=0.0,0.0,0.0
with open("./wankp","w") as f:
	f.write("%s f 1 1 \n"%(arg[0]*arg[1]*arg[2]))
	
for z in range(arg[2]):
	for y in range(arg[1]):
		for x in range(arg[0]):
			with open("./=.kp","a") as f:
				f.write("%s %s %s\n"%(repr(x*xtel).ljust(20),
                repr(y*ytel).ljust(20),repr(z*ztel).ljust(20)))
				print("%s %s %s\n"%(repr(x*xtel).ljust(20),
                repr(y*ytel).ljust(20),repr(z*ztel).ljust(20)))
\end{lstlisting}
\end{latin}
% \end{landscape}
نمونه کد برای زبان جولیا
\begin{latin}
\begin{lstlisting}[style=julia]
#= This is a code sample for the Julia language
(adapted from http://julialang.org) =#
function mandel(z)
    c = z
    maxiter = 80
    for n = 1:maxiter
        if abs(z) > 2
            return n-1
        end
        z = z^2 + c
    end
    return maxiter
end

function helloworld()
    println("Hello, World!") # Bye bye, MATLAB!
end

function randmatstat(t)
    n = 5
    v = zeros(t)
    w = zeros(t)
    for i = 1:t
        a = randn(n,n)
        b = randn(n,n)
        c = randn(n,n)
        d = randn(n,n)
        P = [a b c d]
        Q = [a b; c d]
        v[i] = trace((P.'*P)^4)
        w[i] = trace((Q.'*Q)^4)
    end
    std(v)/mean(v), std(w)/mean(w)
end
\end{lstlisting}
\end{latin}
نمونه کد برای کد متلب
\begin{latin}
 \begin{lstlisting}[style=mymatlab]
  n = 40;
y = randi([500, 600], 1, n);
a = zeros(n,1);

% PARFOR-Loop (no workers)
if matlabpool('size') > 0, matlabpool close, end

p1 = Par(n);

parfor id = 1:n
  Par.tic;
  a(id) = max(svd(rand(y(id))));
  p1(id) = Par.toc;
end

stop(p1);

plot(p1);

% Plot using optional colormap input
% plot(p1,@bone);
 \end{lstlisting}
\end{latin}
نمونه کد برای فرترن
\begin{latin}
 \begin{lstlisting}[style=myfortran]
! Der folgende Fortran-Code ist bei Wikipedia geklaut.
SUBROUTINE test( Argument1, Argument2, Argument3 )
   REAL,              INTENT(IN) :: Argument1
   CHARACTER(LEN= *), INTENT(IN) :: Argument2
   INTEGER,           INTENT(IN), OPTIONAL :: Argument3
   ! This makes sense
END SUBROUTINE
\end{lstlisting}
\end{latin}


