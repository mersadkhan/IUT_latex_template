\fancyhf{}
\chapter*{\vfill چکیده}
%\chapter*{ چکیده}
\addcontentsline{toc}{chapter}{چکیده}
%\begin{center}
%\begin{minipage}{.8\textwidth}
{\fontsize{10}{11}\selectfont
در این قالب پایان‌نامه نمونه‌هایی برای نوشتن یک پایان‌نامه و یا متن در محیط لاتک توضیح داده می شود در ابتدا نصب تکلایو روی سیستم عامل‌های مختلف توضیح داده می‌شود و همچنین تنظیمات مربوط به IDE بیان می شود و در ادامه ابزار‌های پرکاربرد مانند تغییرات فونت،رسم تصویر و شکل و جدول، فرمول‌نویسی، مرجع زنی و فهرست‌های مختلف مانند لغت نامه و اختصارات، نوشتن کد به زبان‌های مختلف و... آمده است. برای مطالعه این فایل لازم است همراه با خوانش فایل \lr{PDF} دستوراتی که منجر به تولید این پایان‌نامه می‌شود  را در فایل کد منبع لاتک آن دنبال کنید. برای نوشتن پایان‌نامه خودتان فقط کافی است جایی را که با نام و عناوین مربوط به استاد راهنما و یا نام نویسنده به صورت متغیر آمده است با عبارات مطلوب خود پرکنید. کلاس این پایان‌نامه طوری طراحی شده است که مقادیر داده شده را طبق یک تنظیم استاندارد جایگذاری می‌کند. از این رو تنها کاری که شما باید در نوشتن پایان‌نامه با این قالب انجام دهید صرفا جایگذاری مقادیر خودتان است. سعی شده است تا جای ممکن حتی در کدهای درون قالب نیز توضیحات مفصلی داده شود.

\keywords{
قالب پایان‌نامه، دانشگاه صنعتی اصفهان، امرکز ابررایانش ملی ایران
 }}
 \newpage
%\end{minipage}
%\end{center}
